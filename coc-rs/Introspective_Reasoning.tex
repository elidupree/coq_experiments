\documentclass{article}

\usepackage{tikz}
\usepackage{amsfonts}
\usepackage{amssymb}
%\usepackage[nointegrals]{wasysym}
\usepackage{mathtools}
\usepackage{mathpartir}
\usepackage{nameref}
\usepackage{xcolor}

\usepackage[english]{babel}
\usepackage[autostyle]{csquotes}
\usepackage{amsmath}
\usepackage{wasysym}
\MakeOuterQuote{"}
\DeclareMathAlphabet{\mathpzc}{OT1}{pzc}{m}{it}

%\usepackage[backend=biber,style=numeric]{biblatex}
%\addbibresource{Introspective_Calculus.bib}

\title{Introspective Reasoning}
\author{Crystal Dendrite*}
\date{\today}

\DeclareMathSymbol{\mlq}{\mathord}{operators}{``}
\DeclareMathSymbol{\mrq}{\mathord}{operators}{`'}

\begin{document}
  \maketitle
  
  \section{Introduction}
  
  How shall a proof system handle paradoxes of self-reference, like Russell's paradox and Girard's paradox?

  To avoid unsoundness, proof-systems typically prevent you from \emph{expressing} self-referential claims.
  However, general-purpose computation can inevitably construct self-referential claims.
  So this protection also denies you the full power of computation.

  Here, we present a simple approach that allows powerful reasoning, even with self-reference.

  \section{The source of unsoundness}

  To understand where paradoxes come from, let's construct an example and see what it takes to break it.

  We don't want our system to prove $\bot$ (the absurd); but let's suppose we are allowed to express claims that "say anything we want", including \emph{referring} to themselves and $\bot$. In particular, let's assume we can construct a paradoxical statement $P$, where:

  \begin{align*}
    P &\equiv (P \to \bot) \\
  \end{align*}

  and assume that $\equiv$ means the two sides can always be treated as the same object.

  Now, we need an \textbf{unsound ruleset}, which can produce $\bot$ from nothing. These are enough:

  \begin{align*}
    \label{refl}
    \tag{reflexivity}
    &\vdash (A \to A) \\
    \label{mp}
    \tag{modus ponens}
    (A \to B), A &\vdash B \\
    \label{unsoundrule}
    \tag{the trickster}
    (C \to (A \to B)), (C \to A) &\vdash (C \to B) \\
  \end{align*}

  \eqref{refl} gets us $P \to P$, which is the same as $P \to (P \to \bot)$.
  This fulfills both premises of \eqref{unsoundrule}, getting us $P \to \bot$.
  Which is the same as $P$, so it fulfills both premises of \eqref{mp}, getting us $\bot$.

  Why is \eqref{unsoundrule} bad? All it says is that you can use \eqref{mp} under a hypothetical $C$.
  
  To see why, let's ask what we \emph{mean} by the relation "$\to$".

  \section{What is implication?}

  We would like to express arbitrary \textbf{systems of reasoning}, which we define by rules that say certain premises go to certain conclusions. This will obey \eqref{refl} and \eqref{trans}:
  \begin{align*}
    \tag{reflexivity}
    &\vdash (A \to A) \\
    \label{trans}
    \tag{transitivity}
    (A \to B), (B \to C) &\vdash (A \to C) \\
  \end{align*}

  But notice that the premises and conclusions of \emph{these} rules have exactly one "$\to$" each.
  If we start putting "$\to$" inside the operands of other "$\to$", we are doing reasoning \emph{about} reasoning: "A particular piece of \emph{reasoning} goes to another."
  And there's more than one way to define which reasoning-about-reasoning to accept.

  [you assume stronger reasoning <-> you can do weaker reasoning]

  ["if A is true, then B must be true" - but we're still in the midst of defining what it means for "A -> B" to be "true", so,]

  [A -> B is possible under certain reasoning - but which reasoning? Let's generalize over kinds of reasoning. conveniently the way to do this is ...]

  In the proposition $A \to B$, where $A$ and $B$ both represent reasoning, one may think of $A$ as the starting-rules and $B$ as derived-rules. If we say:
  \begin{align*}
    \tag{derive reflexivity}
    &\vdash B \to (A \to A) \\
  \end{align*}

  then we cease to acknowledge that \emph{any} system of reasoning can \emph{fail} to derive \eqref{refl}.

  Similarly, a system could assert \emph{any} of its own beliefs as universal truths that can be derived from anything:
  \begin{align*}
    \tag{weakening}
    A &\vdash (B \to A) \\
  \end{align*}

  By including \eqref{mp} in a system, we say: You may reason from any $A$ to all \textbf{consequences} of $A$ (the $B$ where ($A \to B$)).


  %\printbibliography
\end{document}
