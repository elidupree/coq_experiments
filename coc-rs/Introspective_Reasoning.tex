\documentclass{article}

\usepackage{tikz}
\usepackage{amsfonts}
\usepackage{amssymb}
%\usepackage[nointegrals]{wasysym}
\usepackage{mathtools}
\usepackage{mathpartir}
\usepackage{nameref}
\usepackage{xcolor}

\usepackage[english]{babel}
\usepackage[autostyle]{csquotes}
\usepackage{amsmath}
\usepackage{wasysym}
\MakeOuterQuote{"}
\DeclareMathAlphabet{\mathpzc}{OT1}{pzc}{m}{it}

%\usepackage[backend=biber,style=numeric]{biblatex}
%\addbibresource{Introspective_Calculus.bib}

\title{Introspective Reasoning}
\author{Crystal Dendrite*}
\date{\today}

\DeclareMathSymbol{\mlq}{\mathord}{operators}{``}
\DeclareMathSymbol{\mrq}{\mathord}{operators}{`'}
\renewcommand{\emptyset}{\varnothing}

\begin{document}
  \maketitle
  
  \section{Introduction}
  
  How shall a proof system handle paradoxes of self-reference, like Russell's paradox and Girard's paradox?

  To avoid unsoundness, proof-systems typically prevent you from \emph{expressing} self-referential claims.
  However, general-purpose computation can inevitably construct self-referential claims.
  So this protection also denies you the full power of computation.

  Here, we present a simple approach that allows powerful reasoning, even with self-reference.

  \section{The source of unsoundness}

  To understand where paradoxes come from, let's construct an example and see what it takes to break it.

  We don't want our system to prove $\bot$ (the absurd); but let's suppose we are allowed to express claims that "say anything we want", including \emph{referring} to themselves and $\bot$. In particular, let's assume we can construct a paradoxical statement $P$, where:

  \begin{align*}
    P &\equiv (P \to \bot) \\
  \end{align*}

  and assume that $\equiv$ means the two sides can always be treated as the same object.

  Now, we need an \textbf{unsound ruleset}, which can produce $\bot$ from nothing. These are enough:

  \begin{align*}
    \label{refl}
    \tag{reflexivity}
    &\vdash (A \to A) \\
    \label{mp}
    \tag{modus ponens}
    (A \to B), A &\vdash B \\
    \label{unsoundrule}
    \tag{the trickster}
    (C \to (A \to B)), (C \to A) &\vdash (C \to B) \\
  \end{align*}

  \eqref{refl} gets us $P \to P$, which is the same as $P \to (P \to \bot)$.
  This fulfills both premises of \eqref{unsoundrule}, getting us $P \to \bot$.
  Which is the same as $P$, so it fulfills both premises of \eqref{mp}, getting us $\bot$.

  Why is \eqref{unsoundrule} bad? All it says is that you can use \eqref{mp} under a hypothetical $C$.
  
  To see why, let's ask what we \emph{mean} by the relation "$\to$".

  \section{What is implication?}

  We would like to express arbitrary \textbf{systems of reasoning}, which we define by rules that say certain premises go to certain conclusions.
  $(A \to B)$ could represent a potential \emph{rule} \textendash "you may reason from $A$ to $B$."

  Can it also represent a proposition that could be true or false? $(A \to B)$ would be true if you can "objectively" reason from $A$ to $B$.
  But reason \emph{under what rules}?
  We're still in the middle of deciding what rules should be allowed.
  If $\to$ \emph{means} using certain rules, and we haven't decided what those rules are, it's hard to justify whether a rule \emph{about} $\to$ should be allowed.

  We take refuge in "full generality": For \emph{any} rules $R$, we want to be able to express the question of whether $R$ can reason from $A$ to $B$.
  We can express this with a \emph{concrete} relation between rules, where $R \twoheadrightarrow S$ means "the rule(s) $R$ can derive the rule(s) $S$ through nothing but using the rules of $R$ in sequence":
  \begin{align*}
    \label{rrefl}
    \tag{reflexivity}
    &\vdash R \twoheadrightarrow (A \to A) \\
    \label{rtrans}
    \tag{transitivity}
    (R \twoheadrightarrow (A \to B, B \to C)) &\vdash (R \twoheadrightarrow (A \to C)) \\
  \end{align*}

  Note that \emph{under} $\twoheadrightarrow$, the meaning of the relation $\to$ isn't defined by our system \textendash\ it's defined by whatever's in $R$.

  What's the fate of \eqref{mp} in this system?
  Since our top-level propositions are only $\twoheadrightarrow$, those are our only possible conclusions.
  And since we claim that they are defined \emph{exhaustively} by \eqref{rrefl} and \eqref{rtrans}, they can only be based on reasoning from \emph{those exact rules}.
  We can express that ruleset internally; I call it $\mathbb{D}$, for "derivation":

  \begin{align*}
    &\vdash \mathbb{D} \twoheadrightarrow (\emptyset \to (A \to A)) \\
    &\vdash \mathbb{D} \twoheadrightarrow ((R \twoheadrightarrow (A \to B, B \to C)) \to (R \twoheadrightarrow (A \to C))) \\
  \end{align*}

  And then use it to express a version of \eqref{mp}:
  \newcommand{\dto}{\mathbb{D} \twoheadrightarrow}

  \begin{align*}
    \label{mp_deriv}
    \tag{modus ponens for $\twoheadrightarrow$}
    (\dto (A \to B)), A &\vdash B \\
  \end{align*}

  Surely this must be a valid rule!
  We surely think our system doesn't produce any wrong beliefs about $(\dto)$, so anything you could derive with this rule is something you could derive anyway.

  At first this may look \emph{too} easy: it kinda looks like $(\mathbb{D} \twoheadrightarrow (A \to B))$ can just be a stand-in for our previous $(A \to B)$. So now let's construct our paradox again:

  \begin{align*}
    P &\equiv (\dto (P \to \bot)) \\
    \label{unsoundrule_deriv}
    \tag{the trickster for $\twoheadrightarrow$}
    (\dto (C \to (\mathbb{D} \to (A \to B)))), (\dto (C \to A)) &\vdash (\dto (C \to B)) \\
  \end{align*}

  But now we can understand the trickster's claim.
  It says: If a system $C$ can reason from $\mathbb{D}$ to $A \to B$, and also can implement $A$ itself, then surely it implements $B$.
  This is plainly \emph{false}!
  $C$'s rules for $\to$ could simply have gone beyond our definition of $\twoheadrightarrow$.

  This will obey \eqref{refl} and \eqref{trans}:
  \begin{align*}
    \tag{reflexivity}
    &\vdash (A \to A) \\
    \label{trans}
    \tag{transitivity}
    (A \to B), (B \to C) &\vdash (A \to C) \\
    (S \looparrowright (A \to (B \looparrowright C)), (S \looparrowright (A \to B)) &\vdash (S \looparrowright (A \to C)) \\
  \end{align*}

  But notice that the premises and conclusions of \emph{these} rules have exactly one "$\to$" each.
  If we start putting "$\to$" inside the operands of other "$\to$", we are doing reasoning \emph{about} reasoning: "A particular piece of \emph{reasoning} goes to another."
  And there's more than one way to define which reasoning-about-reasoning to accept.

  [you assume stronger reasoning <-> you can do weaker reasoning]

  ["if A is true, then B must be true" - but we're still in the midst of defining what it means for "A -> B" to be "true", so,]

  [A -> B is possible under certain reasoning - but which reasoning? Let's generalize over kinds of reasoning. conveniently the way to do this is ...]

  In the proposition $A \to B$, where $A$ and $B$ both represent reasoning, one may think of $A$ as the starting-rules and $B$ as derived-rules. If we say:
  \begin{align*}
    \tag{derive reflexivity}
    &\vdash B \to (A \to A) \\
  \end{align*}

  then we cease to acknowledge that \emph{any} system of reasoning can \emph{fail} to derive \eqref{refl}.

  Similarly, a system could assert \emph{any} of its own beliefs as universal truths that can be derived from anything:
  \begin{align*}
    \tag{weakening}
    A &\vdash (B \to A) \\
  \end{align*}

  By including \eqref{mp} in a system, we say: You may reason from any $A$ to all \textbf{consequences} of $A$ (the $B$ where ($A \to B$)).


  %\printbibliography
\end{document}
