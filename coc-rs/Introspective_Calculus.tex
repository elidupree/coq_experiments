\documentclass{article}

\usepackage{tikz}
\usepackage{amsfonts}
\usepackage{amssymb}
%\usepackage[nointegrals]{wasysym}
\usepackage{mathtools}
\usepackage{mathpartir}
\usepackage{nameref}
\usepackage{xcolor}

\usepackage[english]{babel}
\usepackage[autostyle]{csquotes}
\usepackage{amsmath}
\usepackage{wasysym}
\MakeOuterQuote{"}

%\usepackage[backend=biber,style=numeric]{biblatex}
%\addbibresource{Introspective_Calculus.bib}

\title{The Introspective Calculus}
\author{Eli Dupree}
\date{\today}

\DeclareMathSymbol{\mlq}{\mathord}{operators}{``}
\DeclareMathSymbol{\mrq}{\mathord}{operators}{`'}

\begin{document}
  \maketitle
  
  \section{Introduction}
  
  This document is a bare-bones explanation of my current definitions of this calculus.
  I plan to later develop it into a full, polished explanation, once I'm more confident in its soundness.
  
  In type theory, the source of paradoxes (like Russell's paradox and Girard's paradox) is this:
  
  \begin{itemize}
    \item A function $F$ takes a parameter $x$ whose type is a function type, and thereby defines \emph{infinite premises} (because $x$ must be correct for every possible value of $x$'s own parameter).
    \item The type derivation for $F$ then uses these infinite premises to derive \emph{infinite conclusions} (the fact that $F$ itself is correct for every possible value of $x$).
    \item By jiggling the infinities, the paradox makes these two infinities line up with each other with a slight offset, so it can pull an instance of False out of the loose end.
  \end{itemize}
  
  Type theories typically prevent this by limiting how you can jiggle the infinities.
  The present work's innovation is to prevent you from making infinite assumptions at all, by adding an unremovable placeholder for every time you apply an abstraction. Thus, the finite-ness of the placeholder stack proves the finite-ness of the proof.
  This allows us to remove the other limitations.
  
  In particular, the source of the name "Introspective Calculus" is that every axiom is represented, and can be reasoned about, within the calculus itself.
  
  Section \ref{fundamentals} (\textit{\nameref{fundamentals}}) gives a formal definition of IC.  

  \section{Fundamentals}\label{fundamentals}

  \subsection{Syntax}
  \begin{align*}
     Atom &:= \mathrm{true} \mid \mathrm{hole} \mid \mathrm{implies} \mid \mathrm{unfoldsto} \mid \mathrm{abstraction} \mid \mathrm{variable}\\
     F &:= Atom \mid (F F)
  \end{align*}

  \subsection{Notations}
  
%  \[ A\ B\ C \dots Y\ Z := ((\dots ((A B) C) \dots Y) Z) \]
  
  \newcommand{\true}{\top}
  \newcommand{\hole}{\circ}
  
  \renewcommand{\implies}{\rightarrow}
  \newcommand{\unfoldsto}{\looparrowright}
  \newcommand{\abstraction}[2]{\Rightarrow\hspace{-0.4em}[#1]\,#2}
  \newcommand{\variable}[1]{{\mathcal{V}_{\mlq\hspace{-0.06em}#1\hspace{-0.06em}\mrq}}}
  \begin{align*}
    \true &:= \mathrm{true}\\
    \hole &:= \mathrm{hole}\\
    (A \implies B) &:= ((\mathrm{implies}\ A) B),\ \operatorname{right-associative}\\
    (A \unfoldsto B) &:= ((\mathrm{unfoldsto}\ A) B)\\
    (\abstraction{A}{B}) &:= ((\mathrm{abstraction}\ A) B)\\
    (\variable{A}) &:= (\mathrm{variable}\ A)\\
  \end{align*}

  \newcommand{\nameabstraction}[2]{#1 \Rightarrow #2}

  Finally, we define the \emph{named form} of abstractions, $(\nameabstraction{A}{B}),\ \operatorname{right-associative}$.
  This is defined as $(\abstraction{L}{B'})$, where both $B'$ and $L$ (the \emph{variable locations}) are copies of $B$ with the following replacements:
  \begin{itemize}
    \item In $L$, replace any subtree that has no usages of $A$ with the atom $\hole$.
    \item In $L$, replace all usages of $A$ with the atom $\true$.
    \item In $B'$, replace all usages of $A$ with the atom $\hole$.
  \end{itemize}

  \newcommand{\here}{\true}

  \subsection{Axiom Definitions}
  \setlength{\jot}{1.4em}
  \begin{gather*}
    \tag{true}
    \inferrule{}{\true}\\
    \tag{modus ponens}
    \inferrule{A\\\\ (A \implies B)}{B}\\
    \tag{weakening}
    \inferrule{B}{A \implies B}\\
    \tag{help, I can't find the name of this rule}
    \inferrule{A \implies B}{(C \implies A)\implies(C \implies B)}\\
    \tag{replace here}
    \inferrule{}{((\abstraction{\here}{B}) C) \unfoldsto C}\\
    \tag{replace not here}
    \inferrule{}{((\abstraction{\hole}{B}) C) \unfoldsto B}\\
    \tag{replace in both}
    \inferrule{
      ((\abstraction{A}{B}) C) \unfoldsto D\\\\
      ((\abstraction{E}{F}) G) \unfoldsto H
    }{
      ((\abstraction{(AE)}{(BF)}) (CG)) \unfoldsto (DH)
    }\\
    \tag{specialization}
    \inferrule{\abstraction{x}{B} \\\\ ((\abstraction{x}{B}) v) \unfoldsto C}{\true \implies C}\\
    \tag{generalization}
    \inferrule{((\abstraction{x}{B})\ \variable{B}) \unfoldsto C\\\\C}{\abstraction{x}{B}}\\
  \end{gather*}

  \subsection{Internal Axioms}

  We now define \emph{internal} versions of the above axioms.
  
  From the internal perspective of the calculus, the above axioms provide truth for any \emph{specific} values of the variables $A, B\dots$, but cannot prove abstractions over those variables.
  So, for each axiom, we also create an internal form, by representing the variables using abstractions, and the premises using implication. For example, (modus ponens) is represented as:

  \begin{equation*}
    \tag{modus ponens, internal}
    \inferrule{}{\nameabstraction{A}{\nameabstraction{B}{A \implies (A \implies B) \implies B}}}\\
  \end{equation*}
  
  One might worry that we would need an internal form of the internal form, and so on indefinitely, but fortunately we do not, because the internal form has no variables or premises.
  
  One axiom (true) does not need an internal form, because it has no variables or premises in the first place. One axiom (generalization) does not need an \emph{external} form, because once the other external axioms are defined, they can derive the external form of generalization from the internal form.

%  \section{Examples}\label{structure}


  %\printbibliography
\end{document}
