\documentclass{article}

\usepackage{tikz}
\usepackage{amsfonts}
\usepackage{amssymb}
%\usepackage[nointegrals]{wasysym}
\usepackage{mathtools}
\usepackage{mathpartir}
\usepackage{nameref}
\usepackage{xcolor}

\usepackage[english]{babel}
\usepackage[autostyle]{csquotes}
\usepackage{amsmath}
\usepackage{wasysym}
\MakeOuterQuote{"}

%\usepackage[backend=biber,style=numeric]{biblatex}
%\addbibresource{Introspective_Calculus.bib}

\title{The Introspective Calculus}
\author{Eli Dupree}
\date{\today}

\DeclareMathSymbol{\mlq}{\mathord}{operators}{``}
\DeclareMathSymbol{\mrq}{\mathord}{operators}{`'}

\begin{document}
  \maketitle
  
  \section{Introduction}
  
  This document is a bare-bones explanation of my current definitions of this calculus.
  I plan to later develop it into a full, polished explanation, once I'm more confident in its soundness.
  
  In type theory, the source of paradoxes (like Russell's paradox and Girard's paradox) is this:
  
  \begin{itemize}
    \item A function $F$ takes a parameter $x$ whose type is a function type, and thereby defines \emph{infinite premises} (because $x$ must be correct for every possible value of $x$'s own parameter).
    \item The type derivation for $F$ then uses these infinite premises to derive \emph{infinite conclusions} (the fact that $F$ itself is correct for every possible value of $x$).
    \item By jiggling the infinities, the paradox makes these two infinities line up with each other with a slight offset, so it can pull an instance of False out of the loose end.
  \end{itemize}
  
  Type theories typically prevent this by limiting how you can jiggle the infinities.
  The present work's innovation is to prevent infinities directly, by adding an unremovable placeholder for every time you create or apply an abstraction, so that the finite-ness of the placeholder stack proves the finite-ness of the proof.
  This allows us to remove the other limitations.
  In particular, the source of the name "Introspective Calculus" is that every rule of IC can be abstracted over within IC.

  The placeholders take the form of well-founded sets, which avoid the Burali-Forti paradox by being subject to the same finite-ness requirements themselves.


  
  Section \ref{fundamentals} (\textit{\nameref{fundamentals}}) gives a formal definition of IC.  

  \section{Fundamentals}\label{fundamentals}

  \subsection{Syntax}
  \begin{align*}
     Atom :=&\ \mathrm{here} \mid \mathrm{hole} \mid \mathrm{implies} \mid \mathrm{unfoldsto} \mid \mathrm{abstraction} \mid \mathrm{variable} \mid \mathrm{clock} \mid\\
      &\ \ \ \ \mathrm{plus} \mid \mathrm{minus} \mid \mathrm{well\_founded\_zero} \mid \mathrm{well\_founded\_successor}\\
     F :=&\ Atom \mid (F F)
  \end{align*}

  \subsection{Notations}
  
%  \[ A\ B\ C \dots Y\ Z := ((\dots ((A B) C) \dots Y) Z) \]
  
  \newcommand{\here}{\bigstar}
  \newcommand{\hole}{\circ}
  \newcommand{\wfz}{0}
  \newcommand{\wfsucc}{\operatorname{\mathrm{S}}}
  
  \renewcommand{\implies}{\rightarrow}
  \newcommand{\wellfounded}{\operatorname{\mathrm{well\_founded}}}
  \newcommand{\unfoldsto}{\looparrowright}
  \newcommand{\abstraction}[2]{\Rightarrow\hspace{-0.4em}[#1]\,#2}
  \newcommand{\variable}[1]{{\mathcal{V}_{\mlq\hspace{-0.06em}#1\hspace{-0.06em}\mrq}}}
  \newcommand{\clocksub}[1]{\text{\clock}_{\!#1}}
  \newcommand{\nameabstraction}[1]{#1 \Rightarrow}

  \begin{align*}
    \here &:= \mathrm{here}\\
    \hole &:= \mathrm{hole}\\
    + &:= \mathrm{plus}\\
    - &:= \mathrm{minus}\\
    \wfz &:= \mathrm{well\_founded\_zero}\\
    \wfsucc &:= \mathrm{well\_founded\_successor}\\
    (A \implies B) &:= ((\mathrm{implies}\ A) B),\ \operatorname{right-associative}\\
    (A \unfoldsto B) &:= ((\mathrm{unfoldsto}\ A) B)\\
    (\abstraction{A}{B}) &:= ((\mathrm{abstraction}\ A) B)\\
    (\variable{A}) &:= (\mathrm{variable}\ A)\\
    \clocksub{A} &:= (\mathrm{clock}\ A)\\
  \end{align*}

  Finally, we define the \emph{named form} of abstractions, $(\nameabstraction{A}B),\ \operatorname{right-associative}$.
  This is defined as $(\abstraction{L}{B'})$, where both $B'$ and $L$ (the \emph{variable locations}) are copies of $B$ with the following replacements:
  \begin{itemize}
    \item In $L$, replace any subtree that has no usages of $A$ with the atom $\hole$.
    \item In $L$, replace all usages of $A$ with the atom $\here$.
    \item In $B'$, replace all usages of $A$ with the atom $\hole$.
  \end{itemize}

  \subsection{Axiom Definitions}
  \setlength{\jot}{1.4em}
  \begin{gather*}
    \tag{modus ponens}\label{mp}
    \inferrule{A\\\\ (A \implies B)}{B}\\
  \end{gather*}
  All remaining axioms are defined as implications ($Premise \implies Premise \dots \implies Conclusion$), which allows them to be reasoned about internally. (The external form can be derived using \eqref{mp}.) However, we still write them as $\frac{Premise\dots}{Conclusion}$ for readability.
  \\\\
  Implication:
  \begin{gather*}
    \tag{weakening}
    \inferrule{B}{A \implies B}\\
    \tag{implication within hypothetical}
    \inferrule{C \implies (A \implies B)}{(C \implies A)\implies(C \implies B)}\\
    \tag{implication within clocks}
    \inferrule{\clocksub{n} (A \implies B)}{(\clocksub{n} A)\implies(\clocksub{n} B)}\\
  \end{gather*}
  Well-foundedness:
  \begin{gather*}
    \tag{successors contain predecessors}
    \inferrule{}{(\wfsucc n) n}\\
    \tag{zero case}
    \inferrule{}{\wellfounded \wfz}\\
    \tag{successor case}
    \inferrule{\wellfounded n}{\wellfounded (\wfsucc n)}\\
    \tag{infinite case}
    \inferrule{\nameabstraction{x}(n x) \implies (\wellfounded x)}{\wellfounded n}\\
  \end{gather*}
  Clocks:
  \begin{gather*}
    \tag{later, now will be earlier}
    \inferrule{A\\\\\wellfounded n}{\clocksub{+n} \clocksub{-n} A}\\
    \tag{later later is more later}
    \inferrule{\clocksub{+o} (\wellfounded n)\\\\no\\\\np\\\\\clocksub{+o}\clocksub{+p}A}{\clocksub{+n} A}\\
    \tag{more earlier is earlier earlier}
    \inferrule{\clocksub{+o} (\wellfounded n)\\\\no\\\\np\\\\\clocksub{-n} A}{\clocksub{-o}\clocksub{-p}A}\\
  \end{gather*}
  Substitution:
  \begin{gather*}
    \tag{replace here}
    \inferrule{}{((\abstraction{\here}{B}) C) \unfoldsto C}\\
    \tag{replace not here}
    \inferrule{}{((\abstraction{\hole}{B}) C) \unfoldsto B}\\
    \tag{replace in both}
    \inferrule{
      ((\abstraction{A}{B}) C) \unfoldsto D\\\\
      ((\abstraction{E}{F}) G) \unfoldsto H
    }{
      ((\abstraction{(AE)}{(BF)}) (CG)) \unfoldsto (DH)
    }\\
  \end{gather*}
  Abstraction:
  \begin{gather*}
    \tag{evaluation of predicates}
    \inferrule{((\abstraction{x}{B})v) \unfoldsto C\\\\C}{(\abstraction{x}{B})v}\\
    \tag{generalization}
    \inferrule{(\abstraction{x}{\clocksub{-\wfz}B})\ \variable{B}}{\abstraction{x}{B}}\\
    \tag{specialization}
    \inferrule{((\abstraction{x}{B}) v) \unfoldsto C\\\\\abstraction{x}{B}}{\clocksub{+\wfz} C}\\
%    \tag{induction}
%    \inferrule{P\ Atom\dots\\\\\nameabstraction{A} \nameabstraction{B} P A \implies P B \implies P (A B)}{P C}\\
  \end{gather*}

  \section{Examples}\label{structure}

  We can define the conventional logical connectives in terms of the above rules:

  \setlength{\jot}{0.4em}
  \begin{align*}
    \tag{false/absurdity}
    \bot &:= (\nameabstraction{P}{P})\\
    \tag{negation}
    \neg P &:= ({P \implies \bot})\\
    \mathrm{not} &:= (\nameabstraction{P} \neg P)\\
    \tag{conjunction}
    P \land Q &:= (\nameabstraction{R} (P \implies Q \implies R) \implies R)\\
    \mathrm{and} &:= (\nameabstraction{P}\nameabstraction{Q} P \land Q)\\
    \tag{disjunction}
    P \lor Q &:= (\nameabstraction{R} (P \implies R) \implies (Q \implies R) \implies R)\\
    \mathrm{or} &:= (\nameabstraction{P}\nameabstraction{Q} P \lor Q)\\
    \tag{existential quantification}
    \exists x, P(x) &:= (\nameabstraction{R} (\nameabstraction{x} P(x) \implies R) \implies R)\\
  \end{align*}

%  And we can prove some tautologies:
%  \begin{align*}
%  (P \land Q)
%  \end{align*}

  %\printbibliography
\end{document}
